\@namedef{1}{Riguardo alle carni sacrificate agli idoli, so che tutti ne abbiamo conoscenza. Ma la conoscenza riempie di orgoglio, mentre l’amore edifica. }%
\@namedef{2}{Se qualcuno crede di conoscere qualcosa, non ha ancora imparato come bisogna conoscere. }%
\@namedef{3}{Chi invece ama Dio, è da lui conosciuto. }%
\@namedef{4}{Riguardo dunque al mangiare le carni sacrificate agli idoli, noi sappiamo che non esiste al mondo alcun idolo e che non c’è alcun dio, se non uno solo. }%
\@namedef{5}{In realtà, anche se vi sono cosiddetti dèi sia nel cielo che sulla terra – e difatti ci sono molti dèi e molti signori –, \par}%
\@namedef{6}{\bpoem{}per noi c’è un solo Dio, il Padre, \epoem\bpoem{}dal quale tutto proviene e noi siamo per lui; \epoem\bpoem{}e un solo Signore, Gesù Cristo, \epoem\bpoem{}in virtù del quale esistono tutte le cose e noi esistiamo grazie a lui. \epoem\poemsep}%
\@namedef{7}{Ma non tutti hanno la conoscenza; alcuni, fino ad ora abituati agli idoli, mangiano le carni come se fossero sacrificate agli idoli, e così la loro coscienza, debole com’è, resta contaminata. }%
\@namedef{8}{Non sarà certo un alimento ad avvicinarci a Dio: se non ne mangiamo, non veniamo a mancare di qualcosa; se ne mangiamo, non ne abbiamo un vantaggio. }%
\@namedef{9}{Badate però che questa vostra libertà non divenga occasione di caduta per i deboli. }%
\@namedef{10}{Se uno infatti vede te, che hai la conoscenza, stare a tavola in un tempio di idoli, la coscienza di quest’uomo debole non sarà forse spinta a mangiare le carni sacrificate agli idoli? }%
\@namedef{11}{Ed ecco, per la tua conoscenza, va in rovina il debole, un fratello per il quale Cristo è morto! \par}%
\@namedef{12}{Peccando così contro i fratelli e ferendo la loro coscienza debole, voi peccate contro Cristo. }%
\@namedef{13}{Per questo, se un cibo scandalizza il mio fratello, non mangerò mai più carne, per non dare scandalo al mio fratello. \par}%
\endinput
