\@namedef{1}{Se parlassi le lingue degli uomini e degli angeli, ma non avessi la carità, sarei come bronzo che rimbomba o come cimbalo che strepita. \par}%
\@namedef{2}{E se avessi il dono della profezia, se conoscessi tutti i misteri e avessi tutta la conoscenza, se possedessi tanta fede da trasportare le montagne, ma non avessi la carità, non sarei nulla. \par}%
\@namedef{3}{E se anche dessi in cibo tutti i miei beni e consegnassi il mio corpo per averne vanto, ma non avessi la carità, a nulla mi servirebbe. \par}%
\@namedef{4}{La carità è magnanima, benevola è la carità; non è invidiosa, non si vanta, non si gonfia d’orgoglio, }%
\@namedef{5}{non manca di rispetto, non cerca il proprio interesse, non si adira, non tiene conto del male ricevuto, }%
\@namedef{6}{non gode dell’ingiustizia ma si rallegra della verità. }%
\@namedef{7}{Tutto scusa, tutto crede, tutto spera, tutto sopporta. \par}%
\@namedef{8}{La carità non avrà mai fine. Le profezie scompariranno, il dono delle lingue cesserà e la conoscenza svanirà. }%
\@namedef{9}{Infatti, in modo imperfetto noi conosciamo e in modo imperfetto profetizziamo. }%
\@namedef{10}{Ma quando verrà ciò che è perfetto, quello che è imperfetto scomparirà. }%
\@namedef{11}{Quand’ero bambino, parlavo da bambino, pensavo da bambino, ragionavo da bambino. Divenuto uomo, ho eliminato ciò che è da bambino. \par}%
\@namedef{12}{Adesso noi vediamo in modo confuso, come in uno specchio; allora invece vedremo faccia a faccia. Adesso conosco in modo imperfetto, ma allora conoscerò perfettamente, come anch’io sono conosciuto. }%
\@namedef{13}{Ora dunque rimangono queste tre cose: la fede, la speranza e la carità. Ma la più grande di tutte è la carità! \par}%
\endinput
