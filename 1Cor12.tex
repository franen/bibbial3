\@namedef{1}{Riguardo ai doni dello Spirito, fratelli, non voglio lasciarvi nell’ignoranza. \par}%
\@namedef{2}{Voi sapete infatti che, quando eravate pagani, vi lasciavate trascinare senza alcun controllo verso gli idoli muti. }%
\@namedef{3}{Perciò io vi dichiaro: nessuno che parli sotto l’azione dello Spirito di Dio può dire: «Gesù è anàtema!»; e nessuno può dire: «Gesù è Signore!», se non sotto l’azione dello Spirito Santo. \par}%
\@namedef{4}{Vi sono diversi carismi, ma uno solo è lo Spirito; }%
\@namedef{5}{vi sono diversi ministeri, ma uno solo è il Signore; }%
\@namedef{6}{vi sono diverse attività, ma uno solo è Dio, che opera tutto in tutti. }%
\@namedef{7}{A ciascuno è data una manifestazione particolare dello Spirito per il bene comune: }%
\@namedef{8}{a uno infatti, per mezzo dello Spirito, viene dato il linguaggio di sapienza; a un altro invece, dallo stesso Spirito, il linguaggio di conoscenza; }%
\@namedef{9}{a uno, nello stesso Spirito, la fede; a un altro, nell’unico Spirito, il dono delle guarigioni; }%
\@namedef{10}{a uno il potere dei miracoli; a un altro il dono della profezia; a un altro il dono di discernere gli spiriti; a un altro la varietà delle lingue; a un altro l’interpretazione delle lingue. }%
\@namedef{11}{Ma tutte queste cose le opera l’unico e medesimo Spirito, distribuendole a ciascuno come vuole. \par}%
\@namedef{12}{Come infatti il corpo è uno solo e ha molte membra, e tutte le membra del corpo, pur essendo molte, sono un corpo solo, così anche il Cristo. }%
\@namedef{13}{Infatti noi tutti siamo stati battezzati mediante un solo Spirito in un solo corpo, Giudei o Greci, schiavi o liberi; e tutti siamo stati dissetati da un solo Spirito. \par}%
\@namedef{14}{E infatti il corpo non è formato da un membro solo, ma da molte membra. \par}%
\@namedef{15}{Se il piede dicesse: «Poiché non sono mano, non appartengo al corpo», non per questo non farebbe parte del corpo. }%
\@namedef{16}{E se l’orecchio dicesse: «Poiché non sono occhio, non appartengo al corpo», non per questo non farebbe parte del corpo. }%
\@namedef{17}{Se tutto il corpo fosse occhio, dove sarebbe l’udito? Se tutto fosse udito, dove sarebbe l’odorato? }%
\@namedef{18}{Ora, invece, Dio ha disposto le membra del corpo in modo distinto, come egli ha voluto. }%
\@namedef{19}{Se poi tutto fosse un membro solo, dove sarebbe il corpo? \par}%
\@namedef{20}{Invece molte sono le membra, ma uno solo è il corpo. }%
\@namedef{21}{Non può l’occhio dire alla mano: «Non ho bisogno di te»; oppure la testa ai piedi: «Non ho bisogno di voi». \par}%
\@namedef{22}{Anzi proprio le membra del corpo che sembrano più deboli sono le più necessarie; }%
\@namedef{23}{e le parti del corpo che riteniamo meno onorevoli le circondiamo di maggiore rispetto, e quelle indecorose sono trattate con maggiore decenza, }%
\@namedef{24}{mentre quelle decenti non ne hanno bisogno. Ma Dio ha disposto il corpo conferendo maggiore onore a ciò che non ne ha, }%
\@namedef{25}{perché nel corpo non vi sia divisione, ma anzi le varie membra abbiano cura le une delle altre. }%
\@namedef{26}{Quindi se un membro soffre, tutte le membra soffrono insieme; e se un membro è onorato, tutte le membra gioiscono con lui. \par}%
\@namedef{27}{Ora voi siete corpo di Cristo e, ognuno secondo la propria parte, sue membra. }%
\@namedef{28}{Alcuni perciò Dio li ha posti nella Chiesa in primo luogo come apostoli, in secondo luogo come profeti, in terzo luogo come maestri; poi ci sono i miracoli, quindi il dono delle guarigioni, di assistere, di governare, di parlare varie lingue. \par}%
\@namedef{29}{Sono forse tutti apostoli? Tutti profeti? Tutti maestri? Tutti fanno miracoli? \par}%
\@namedef{30}{Tutti possiedono il dono delle guarigioni? Tutti parlano lingue? Tutti le interpretano? }%
\@namedef{31}{Desiderate invece intensamente i carismi più grandi. E allora, vi mostro la via più sublime. \par}%
\endinput
