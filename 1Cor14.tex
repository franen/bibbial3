\@namedef{1}{Aspirate alla carità. Desiderate intensamente i doni dello Spirito, soprattutto la profezia. }%
\@namedef{2}{Chi infatti parla con il dono delle lingue non parla agli uomini ma a Dio poiché, mentre dice per ispirazione cose misteriose, nessuno comprende. }%
\@namedef{3}{Chi profetizza, invece, parla agli uomini per loro edificazione, esortazione e conforto. \par}%
\@namedef{4}{Chi parla con il dono delle lingue edifica se stesso, chi profetizza edifica l’assemblea. }%
\@namedef{5}{Vorrei vedervi tutti parlare con il dono delle lingue, ma preferisco che abbiate il dono della profezia. In realtà colui che profetizza è più grande di colui che parla con il dono delle lingue, a meno che le interpreti, perché l’assemblea ne riceva edificazione. \par}%
\@namedef{6}{E ora, fratelli, supponiamo che io venga da voi parlando con il dono delle lingue. In che cosa potrei esservi utile, se non vi comunicassi una rivelazione o una conoscenza o una profezia o un insegnamento? }%
\@namedef{7}{Ad esempio: se gli oggetti inanimati che emettono un suono, come il flauto o la cetra, non producono i suoni distintamente, in che modo si potrà distinguere ciò che si suona col flauto da ciò che si suona con la cetra? }%
\@namedef{8}{E se la tromba emette un suono confuso, chi si preparerà alla battaglia? }%
\@namedef{9}{Così anche voi, se non pronunciate parole chiare con la lingua, come si potrà comprendere ciò che andate dicendo? Parlereste al vento! \par}%
\@namedef{10}{Chissà quante varietà di lingue vi sono nel mondo e nulla è senza un proprio linguaggio. }%
\@namedef{11}{Ma se non ne conosco il senso, per colui che mi parla sono uno straniero, e chi mi parla è uno straniero per me. \par}%
\@namedef{12}{Così anche voi, poiché desiderate i doni dello Spirito, cercate di averne in abbondanza, per l’edificazione della comunità. }%
\@namedef{13}{Perciò chi parla con il dono delle lingue, preghi di saperle interpretare. }%
\@namedef{14}{Quando infatti prego con il dono delle lingue, il mio spirito prega, ma la mia intelligenza rimane senza frutto. }%
\@namedef{15}{Che fare dunque? Pregherò con lo spirito, ma pregherò anche con l’intelligenza; canterò con lo spirito, ma canterò anche con l’intelligenza. }%
\@namedef{16}{Altrimenti, se tu dai lode a Dio soltanto con lo spirito, in che modo colui che sta fra i non iniziati potrebbe dire l’Amen al tuo ringraziamento, dal momento che non capisce quello che dici? }%
\@namedef{17}{Tu, certo, fai un bel ringraziamento, ma l’altro non viene edificato. }%
\@namedef{18}{Grazie a Dio, io parlo con il dono delle lingue più di tutti voi; }%
\@namedef{19}{ma in assemblea preferisco dire cinque parole con la mia intelligenza per istruire anche gli altri, piuttosto che diecimila parole con il dono delle lingue. \par}%
\@namedef{20}{Fratelli, non comportatevi da bambini nei giudizi. Quanto a malizia, siate bambini, ma quanto a giudizi, comportatevi da uomini maturi. }%
\@namedef{21}{\bpoem{}Sta scritto nella Legge: In altre lingue e con labbra di stranieri \epoem\bpoem{}parlerò a questo popolo, \epoem\bpoem{}ma neanche così mi ascolteranno, \epoem\bpoem{}dice il Signore. }%
\@namedef{22}{Quindi le lingue non sono un segno per quelli che credono, ma per quelli che non credono, mentre la profezia non è per quelli che non credono, ma per quelli che credono. }%
\@namedef{23}{Quando si raduna tutta la comunità nello stesso luogo, se tutti parlano con il dono delle lingue e sopraggiunge qualche non iniziato o non credente, non dirà forse che siete pazzi? }%
\@namedef{24}{Se invece tutti profetizzano e sopraggiunge qualche non credente o non iniziato, verrà da tutti convinto del suo errore e da tutti giudicato, }%
\@namedef{25}{i segreti del suo cuore saranno manifestati e così, prostrandosi a terra, adorerà Dio, proclamando: Dio è veramente fra voi! \par}%
\@namedef{26}{Che fare dunque, fratelli? Quando vi radunate, uno ha un salmo, un altro ha un insegnamento; uno ha una rivelazione, uno ha il dono delle lingue, un altro ha quello di interpretarle: tutto avvenga per l’edificazione. }%
\@namedef{27}{Quando si parla con il dono delle lingue, siano in due, o al massimo in tre, a parlare, uno alla volta, e vi sia uno che faccia da interprete. }%
\@namedef{28}{Se non vi è chi interpreta, ciascuno di loro taccia nell’assemblea e parli solo a se stesso e a Dio. }%
\@namedef{29}{I profeti parlino in due o tre e gli altri giudichino. }%
\@namedef{30}{Ma se poi uno dei presenti riceve una rivelazione, il primo taccia: }%
\@namedef{31}{uno alla volta, infatti, potete tutti profetare, perché tutti possano imparare ed essere esortati. }%
\@namedef{32}{Le ispirazioni dei profeti sono sottomesse ai profeti, }%
\@namedef{33}{perché Dio non è un Dio di disordine, ma di pace. Come in tutte le comunità dei santi, }%
\@namedef{34}{le donne nelle assemblee tacciano perché non è loro permesso parlare; stiano invece sottomesse, come dice anche la Legge. }%
\@namedef{35}{Se vogliono imparare qualche cosa, interroghino a casa i loro mariti, perché è sconveniente per una donna parlare in assemblea. \par}%
\@namedef{36}{Da voi, forse, è partita la parola di Dio? O è giunta soltanto a voi? }%
\@namedef{37}{Chi ritiene di essere profeta o dotato di doni dello Spirito, deve riconoscere che quanto vi scrivo è comando del Signore. }%
\@namedef{38}{Se qualcuno non lo riconosce, neppure lui viene riconosciuto. }%
\@namedef{39}{Dunque, fratelli miei, desiderate intensamente la profezia e, quanto al parlare con il dono delle lingue, non impeditelo. }%
\@namedef{40}{Tutto però avvenga decorosamente e con ordine. \par}%
\endinput
