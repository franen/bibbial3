\@namedef{1}{Non sono forse libero, io? Non sono forse un apostolo? Non ho veduto Gesù, Signore nostro? E non siete voi la mia opera nel Signore? }%
\@namedef{2}{Anche se non sono apostolo per altri, almeno per voi lo sono; voi siete nel Signore il sigillo del mio apostolato. }%
\@namedef{3}{La mia difesa contro quelli che mi accusano è questa: }%
\@namedef{4}{non abbiamo forse il diritto di mangiare e di bere? }%
\@namedef{5}{Non abbiamo il diritto di portare con noi una donna credente, come fanno anche gli altri apostoli e i fratelli del Signore e Cefa? \par}%
\@namedef{6}{Oppure soltanto io e Bàrnaba non abbiamo il diritto di non lavorare? \par}%
\@namedef{7}{E chi mai presta servizio militare a proprie spese? Chi pianta una vigna senza mangiarne il frutto? Chi fa pascolare un gregge senza cibarsi del latte del gregge? \par}%
\@namedef{8}{Io non dico questo da un punto di vista umano; è la Legge che dice così. }%
\@namedef{9}{Nella legge di Mosè infatti sta scritto: Non metterai la museruola al bue che trebbia. Forse Dio si prende cura dei buoi? }%
\@namedef{10}{Oppure lo dice proprio per noi? Certamente fu scritto per noi. Poiché colui che ara, deve arare sperando, e colui che trebbia, trebbiare nella speranza di avere la sua parte. }%
\@namedef{11}{Se noi abbiamo seminato in voi beni spirituali, è forse gran cosa se raccoglieremo beni materiali? }%
\@namedef{12}{Se altri hanno tale diritto su di voi, noi non l’abbiamo di più? Noi però non abbiamo voluto servirci di questo diritto, ma tutto sopportiamo per non mettere ostacoli al vangelo di Cristo. }%
\@namedef{13}{Non sapete che quelli che celebrano il culto, dal culto traggono il vitto, e quelli che servono all’altare, dall’altare ricevono la loro parte? }%
\@namedef{14}{Così anche il Signore ha disposto che quelli che annunciano il Vangelo vivano del Vangelo. \par}%
\@namedef{15}{Io invece non mi sono avvalso di alcuno di questi diritti, né ve ne scrivo perché si faccia in tal modo con me; preferirei piuttosto morire. Nessuno mi toglierà questo vanto! }%
\@namedef{16}{Infatti annunciare il Vangelo non è per me un vanto, perché è una necessità che mi si impone: guai a me se non annuncio il Vangelo! }%
\@namedef{17}{Se lo faccio di mia iniziativa, ho diritto alla ricompensa; ma se non lo faccio di mia iniziativa, è un incarico che mi è stato affidato. }%
\@namedef{18}{Qual è dunque la mia ricompensa? Quella di annunciare gratuitamente il Vangelo senza usare il diritto conferitomi dal Vangelo. \par}%
\@namedef{19}{Infatti, pur essendo libero da tutti, mi sono fatto servo di tutti per guadagnarne il maggior numero: }%
\@namedef{20}{mi sono fatto come Giudeo per i Giudei, per guadagnare i Giudei. Per coloro che sono sotto la Legge – pur non essendo io sotto la Legge – mi sono fatto come uno che è sotto la Legge, allo scopo di guadagnare coloro che sono sotto la Legge. }%
\@namedef{21}{Per coloro che non hanno Legge – pur non essendo io senza la legge di Dio, anzi essendo nella legge di Cristo – mi sono fatto come uno che è senza Legge, allo scopo di guadagnare coloro che sono senza Legge. }%
\@namedef{22}{Mi sono fatto debole per i deboli, per guadagnare i deboli; mi sono fatto tutto per tutti, per salvare a ogni costo qualcuno. }%
\@namedef{23}{Ma tutto io faccio per il Vangelo, per diventarne partecipe anch’io. \par}%
\@namedef{24}{Non sapete che, nelle corse allo stadio, tutti corrono, ma uno solo conquista il premio? Correte anche voi in modo da conquistarlo! }%
\@namedef{25}{Però ogni atleta è disciplinato in tutto; essi lo fanno per ottenere una corona che appassisce, noi invece una che dura per sempre. }%
\@namedef{26}{Io dunque corro, ma non come chi è senza mèta; faccio pugilato, ma non come chi batte l’aria; }%
\@namedef{27}{anzi tratto duramente il mio corpo e lo riduco in schiavitù, perché non succeda che, dopo avere predicato agli altri, io stesso venga squalificato. \par}%
\endinput
