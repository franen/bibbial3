\@namedef{1}{Paolo, chiamato a essere apostolo di Cristo Gesù per volontà di Dio, e il fratello Sòstene, }%
\@namedef{2}{alla Chiesa di Dio che è a Corinto, a coloro che sono stati santificati in Cristo Gesù, santi per chiamata, insieme a tutti quelli che in ogni luogo invocano il nome del Signore nostro Gesù Cristo, Signore nostro e loro: }%
\@namedef{3}{grazia a voi e pace da Dio Padre nostro e dal Signore Gesù Cristo! \par}%
\@namedef{4}{Rendo grazie continuamente al mio Dio per voi, a motivo della grazia di Dio che vi è stata data in Cristo Gesù, }%
\@namedef{5}{perché in lui siete stati arricchiti di tutti i doni, quelli della parola e quelli della conoscenza. }%
\@namedef{6}{La testimonianza di Cristo si è stabilita tra voi così saldamente }%
\@namedef{7}{che non manca più alcun carisma a voi, che aspettate la manifestazione del Signore nostro Gesù Cristo. }%
\@namedef{8}{Egli vi renderà saldi sino alla fine, irreprensibili nel giorno del Signore nostro Gesù Cristo. }%
\@namedef{9}{Degno di fede è Dio, dal quale siete stati chiamati alla comunione con il Figlio suo Gesù Cristo, Signore nostro! \par}%
\@namedef{10}{Vi esorto pertanto, fratelli, per il nome del Signore nostro Gesù Cristo, a essere tutti unanimi nel parlare, perché non vi siano divisioni tra voi, ma siate in perfetta unione di pensiero e di sentire. }%
\@namedef{11}{Infatti a vostro riguardo, fratelli, mi è stato segnalato dai familiari di Cloe che tra voi vi sono discordie. }%
\@namedef{12}{Mi riferisco al fatto che ciascuno di voi dice: «Io sono di Paolo», «Io invece sono di Apollo», «Io invece di Cefa», «E io di Cristo». \par}%
\@namedef{13}{È forse diviso il Cristo? Paolo è stato forse crocifisso per voi? O siete stati battezzati nel nome di Paolo? }%
\@namedef{14}{Ringrazio Dio di non avere battezzato nessuno di voi, eccetto Crispo e Gaio, }%
\@namedef{15}{perché nessuno possa dire che siete stati battezzati nel mio nome. }%
\@namedef{16}{Ho battezzato, è vero, anche la famiglia di Stefanàs, ma degli altri non so se io abbia battezzato qualcuno. }%
\@namedef{17}{Cristo infatti non mi ha mandato a battezzare, ma ad annunciare il Vangelo, non con sapienza di parola, perché non venga resa vana la croce di Cristo. \par}%
\@namedef{18}{La parola della croce infatti è stoltezza per quelli che si perdono, ma per quelli che si salvano, ossia per noi, è potenza di Dio. }%
\@namedef{19}{\bpoem{}Sta scritto infatti: Distruggerò la sapienza dei sapienti \epoem\bpoem{}e annullerò l’intelligenza degli intelligenti. \epoem\poemsep}%
\@namedef{20}{Dov’è il sapiente? Dov’è il dotto? Dov’è il sottile ragionatore di questo mondo? Dio non ha forse dimostrato stolta la sapienza del mondo? }%
\@namedef{21}{Poiché infatti, nel disegno sapiente di Dio, il mondo, con tutta la sua sapienza, non ha conosciuto Dio, è piaciuto a Dio salvare i credenti con la stoltezza della predicazione. }%
\@namedef{22}{Mentre i Giudei chiedono segni e i Greci cercano sapienza, }%
\@namedef{23}{noi invece annunciamo Cristo crocifisso: scandalo per i Giudei e stoltezza per i pagani; }%
\@namedef{24}{ma per coloro che sono chiamati, sia Giudei che Greci, Cristo è potenza di Dio e sapienza di Dio. }%
\@namedef{25}{Infatti ciò che è stoltezza di Dio è più sapiente degli uomini, e ciò che è debolezza di Dio è più forte degli uomini. \par}%
\@namedef{26}{Considerate infatti la vostra chiamata, fratelli: non ci sono fra voi molti sapienti dal punto di vista umano, né molti potenti, né molti nobili. }%
\@namedef{27}{Ma quello che è stolto per il mondo, Dio lo ha scelto per confondere i sapienti; quello che è debole per il mondo, Dio lo ha scelto per confondere i forti; }%
\@namedef{28}{quello che è ignobile e disprezzato per il mondo, quello che è nulla, Dio lo ha scelto per ridurre al nulla le cose che sono, }%
\@namedef{29}{perché nessuno possa vantarsi di fronte a Dio. }%
\@namedef{30}{Grazie a lui voi siete in Cristo Gesù, il quale per noi è diventato sapienza per opera di Dio, giustizia, santificazione e redenzione, }%
\@namedef{31}{perché, come sta scritto, chi si vanta, si vanti nel Signore. \par}%
\endinput
