\@namedef{1}{Diventate miei imitatori, come io lo sono di Cristo. \par}%
\@namedef{2}{Vi lodo perché in ogni cosa vi ricordate di me e conservate le tradizioni così come ve le ho trasmesse. }%
\@namedef{3}{Voglio però che sappiate che di ogni uomo il capo è Cristo, e capo della donna è l’uomo, e capo di Cristo è Dio. }%
\@namedef{4}{Ogni uomo che prega o profetizza con il capo coperto, manca di riguardo al proprio capo. }%
\@namedef{5}{Ma ogni donna che prega o profetizza a capo scoperto, manca di riguardo al proprio capo, perché è come se fosse rasata. }%
\@namedef{6}{Se dunque una donna non vuole coprirsi, si tagli anche i capelli! Ma se è vergogna per una donna tagliarsi i capelli o radersi, allora si copra. \par}%
\@namedef{7}{L’uomo non deve coprirsi il capo, perché egli è immagine e gloria di Dio; la donna invece è gloria dell’uomo. }%
\@namedef{8}{E infatti non è l’uomo che deriva dalla donna, ma la donna dall’uomo; }%
\@namedef{9}{né l’uomo fu creato per la donna, ma la donna per l’uomo. \par}%
\@namedef{10}{Per questo la donna deve avere sul capo un segno di autorità a motivo degli angeli. }%
\@namedef{11}{Tuttavia, nel Signore, né la donna è senza l’uomo, né l’uomo è senza la donna. }%
\@namedef{12}{Come infatti la donna deriva dall’uomo, così l’uomo ha vita dalla donna; tutto poi proviene da Dio. }%
\@namedef{13}{Giudicate voi stessi: è conveniente che una donna preghi Dio col capo scoperto? }%
\@namedef{14}{Non è forse la natura stessa a insegnarci che è indecoroso per l’uomo lasciarsi crescere i capelli, }%
\@namedef{15}{mentre è una gloria per la donna lasciarseli crescere? La lunga capigliatura le è stata data a modo di velo. }%
\@namedef{16}{Se poi qualcuno ha il gusto della contestazione, noi non abbiamo questa consuetudine e neanche le Chiese di Dio. \par}%
\@namedef{17}{Mentre vi do queste istruzioni, non posso lodarvi, perché vi riunite insieme non per il meglio, ma per il peggio. }%
\@namedef{18}{Innanzi tutto sento dire che, quando vi radunate in assemblea, vi sono divisioni tra voi, e in parte lo credo. }%
\@namedef{19}{È necessario infatti che sorgano fazioni tra voi, perché in mezzo a voi si manifestino quelli che hanno superato la prova. }%
\@namedef{20}{Quando dunque vi radunate insieme, il vostro non è più un mangiare la cena del Signore. }%
\@namedef{21}{Ciascuno infatti, quando siete a tavola, comincia a prendere il proprio pasto e così uno ha fame, l’altro è ubriaco. }%
\@namedef{22}{Non avete forse le vostre case per mangiare e per bere? O volete gettare il disprezzo sulla Chiesa di Dio e umiliare chi non ha niente? Che devo dirvi? Lodarvi? In questo non vi lodo! \par}%
\@namedef{23}{Io, infatti, ho ricevuto dal Signore quello che a mia volta vi ho trasmesso: il Signore Gesù, nella notte in cui veniva tradito, prese del pane }%
\@namedef{24}{e, dopo aver reso grazie, lo spezzò e disse: «Questo è il mio corpo, che è per voi; fate questo in memoria di me». }%
\@namedef{25}{Allo stesso modo, dopo aver cenato, prese anche il calice, dicendo: «Questo calice è la nuova alleanza nel mio sangue; fate questo, ogni volta che ne bevete, in memoria di me». }%
\@namedef{26}{Ogni volta infatti che mangiate questo pane e bevete al calice, voi annunciate la morte del Signore, finché egli venga. }%
\@namedef{27}{Perciò chiunque mangia il pane o beve al calice del Signore in modo indegno, sarà colpevole verso il corpo e il sangue del Signore. }%
\@namedef{28}{Ciascuno, dunque, esamini se stesso e poi mangi del pane e beva dal calice; }%
\@namedef{29}{perché chi mangia e beve senza riconoscere il corpo del Signore, mangia e beve la propria condanna. }%
\@namedef{30}{È per questo che tra voi ci sono molti ammalati e infermi, e un buon numero sono morti. }%
\@namedef{31}{Se però ci esaminassimo attentamente da noi stessi, non saremmo giudicati; }%
\@namedef{32}{quando poi siamo giudicati dal Signore, siamo da lui ammoniti per non essere condannati insieme con il mondo. \par}%
\@namedef{33}{Perciò, fratelli miei, quando vi radunate per la cena, aspettatevi gli uni gli altri. }%
\@namedef{34}{E se qualcuno ha fame, mangi a casa, perché non vi raduniate a vostra condanna. Quanto alle altre cose, le sistemerò alla mia venuta. \par}%
\endinput
