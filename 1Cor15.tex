\@namedef{1}{Vi proclamo poi, fratelli, il Vangelo che vi ho annunciato e che voi avete ricevuto, nel quale restate saldi }%
\@namedef{2}{e dal quale siete salvati, se lo mantenete come ve l’ho annunciato. A meno che non abbiate creduto invano! \par}%
\@namedef{3}{\bpoem{}A voi infatti ho trasmesso, anzitutto, quello che anch’io ho ricevuto, cioè che Cristo morì per i nostri peccati secondo le Scritture \epoem\bpoem{}e che }%
\@namedef{4}{\bpoem{}fu sepolto \epoem\bpoem{}e che è risorto il terzo giorno secondo le Scritture \epoem\poemsep}%
\@namedef{5}{e che apparve a Cefa e quindi ai Dodici. \par}%
\@namedef{6}{In seguito apparve a più di cinquecento fratelli in una sola volta: la maggior parte di essi vive ancora, mentre alcuni sono morti. }%
\@namedef{7}{Inoltre apparve a Giacomo, e quindi a tutti gli apostoli. }%
\@namedef{8}{Ultimo fra tutti apparve anche a me come a un aborto. }%
\@namedef{9}{Io infatti sono il più piccolo tra gli apostoli e non sono degno di essere chiamato apostolo perché ho perseguitato la Chiesa di Dio. }%
\@namedef{10}{Per grazia di Dio, però, sono quello che sono, e la sua grazia in me non è stata vana. Anzi, ho faticato più di tutti loro, non io però, ma la grazia di Dio che è con me. }%
\@namedef{11}{Dunque, sia io che loro, così predichiamo e così avete creduto. \par}%
\@namedef{12}{Ora, se si annuncia che Cristo è risorto dai morti, come possono dire alcuni tra voi che non vi è risurrezione dei morti? }%
\@namedef{13}{Se non vi è risurrezione dei morti, neanche Cristo è risorto! }%
\@namedef{14}{Ma se Cristo non è risorto, vuota allora è la nostra predicazione, vuota anche la vostra fede. }%
\@namedef{15}{Noi, poi, risultiamo falsi testimoni di Dio, perché contro Dio abbiamo testimoniato che egli ha risuscitato il Cristo mentre di fatto non lo ha risuscitato, se è vero che i morti non risorgono. }%
\@namedef{16}{Se infatti i morti non risorgono, neanche Cristo è risorto; }%
\@namedef{17}{ma se Cristo non è risorto, vana è la vostra fede e voi siete ancora nei vostri peccati. }%
\@namedef{18}{Perciò anche quelli che sono morti in Cristo sono perduti. }%
\@namedef{19}{Se noi abbiamo avuto speranza in Cristo soltanto per questa vita, siamo da commiserare più di tutti gli uomini. \par}%
\@namedef{20}{Ora, invece, Cristo è risorto dai morti, primizia di coloro che sono morti. \par}%
\@namedef{21}{Perché, se per mezzo di un uomo venne la morte, per mezzo di un uomo verrà anche la risurrezione dei morti. }%
\@namedef{22}{Come infatti in Adamo tutti muoiono, così in Cristo tutti riceveranno la vita. }%
\@namedef{23}{Ognuno però al suo posto: prima Cristo, che è la primizia; poi, alla sua venuta, quelli che sono di Cristo. }%
\@namedef{24}{Poi sarà la fine, quando egli consegnerà il regno a Dio Padre, dopo avere ridotto al nulla ogni Principato e ogni Potenza e Forza. }%
\@namedef{25}{È necessario infatti che egli regni finché non abbia posto tutti i nemici sotto i suoi piedi. \par}%
\@namedef{26}{L’ultimo nemico a essere annientato sarà la morte, }%
\@namedef{27}{perché ogni cosa ha posto sotto i suoi piedi. Però, quando dice che ogni cosa è stata sottoposta, è chiaro che si deve eccettuare Colui che gli ha sottomesso ogni cosa. }%
\@namedef{28}{E quando tutto gli sarà stato sottomesso, anch’egli, il Figlio, sarà sottomesso a Colui che gli ha sottomesso ogni cosa, perché Dio sia tutto in tutti. \par}%
\@namedef{29}{Altrimenti, che cosa faranno quelli che si fanno battezzare per i morti? Se davvero i morti non risorgono, perché si fanno battezzare per loro? }%
\@namedef{30}{E perché noi ci esponiamo continuamente al pericolo? }%
\@namedef{31}{Ogni giorno io vado incontro alla morte, come è vero che voi, fratelli, siete il mio vanto in Cristo Gesù, nostro Signore! }%
\@namedef{32}{Se soltanto per ragioni umane io avessi combattuto a Èfeso contro le belve, a che mi gioverebbe? Se i morti non risorgono, mangiamo e beviamo, perché domani moriremo. }%
\@namedef{33}{Non lasciatevi ingannare: «Le cattive compagnie corrompono i buoni costumi». }%
\@namedef{34}{Tornate in voi stessi, come è giusto, e non peccate! Alcuni infatti dimostrano di non conoscere Dio; ve lo dico a vostra vergogna. \par}%
\@namedef{35}{Ma qualcuno dirà: «Come risorgono i morti? Con quale corpo verranno?». \par}%
\@namedef{36}{Stolto! Ciò che tu semini non prende vita, se prima non muore. }%
\@namedef{37}{Quanto a ciò che semini, non semini il corpo che nascerà, ma un semplice chicco di grano o di altro genere. }%
\@namedef{38}{E Dio gli dà un corpo come ha stabilito, e a ciascun seme il proprio corpo. \par}%
\@namedef{39}{Non tutti i corpi sono uguali: altro è quello degli uomini e altro quello degli animali; altro quello degli uccelli e altro quello dei pesci. }%
\@namedef{40}{Vi sono corpi celesti e corpi terrestri, ma altro è lo splendore dei corpi celesti, altro quello dei corpi terrestri. }%
\@namedef{41}{Altro è lo splendore del sole, altro lo splendore della luna e altro lo splendore delle stelle. Ogni stella infatti differisce da un’altra nello splendore. \par}%
\@namedef{42}{Così anche la risurrezione dei morti: è seminato nella corruzione, risorge nell’incorruttibilità; }%
\@namedef{43}{è seminato nella miseria, risorge nella gloria; è seminato nella debolezza, risorge nella potenza; }%
\@namedef{44}{è seminato corpo animale, risorge corpo spirituale. Se c’è un corpo animale, vi è anche un corpo spirituale. Sta scritto infatti che }%
\@namedef{45}{il primo uomo, Adamo, divenne un essere vivente, ma l’ultimo Adamo divenne spirito datore di vita. }%
\@namedef{46}{Non vi fu prima il corpo spirituale, ma quello animale, e poi lo spirituale. }%
\@namedef{47}{Il primo uomo, tratto dalla terra, è fatto di terra; il secondo uomo viene dal cielo. }%
\@namedef{48}{Come è l’uomo terreno, così sono quelli di terra; e come è l’uomo celeste, così anche i celesti. }%
\@namedef{49}{E come eravamo simili all’uomo terreno, così saremo simili all’uomo celeste. }%
\@namedef{50}{Vi dico questo, o fratelli: carne e sangue non possono ereditare il regno di Dio, né ciò che si corrompe può ereditare l’incorruttibilità. \par}%
\@namedef{51}{Ecco, io vi annuncio un mistero: noi tutti non moriremo, ma tutti saremo trasformati, }%
\@namedef{52}{in un istante, in un batter d’occhio, al suono dell’ultima tromba. Essa infatti suonerà e i morti risorgeranno incorruttibili e noi saremo trasformati. }%
\@namedef{53}{È necessario infatti che questo corpo corruttibile si vesta d’incorruttibilità e questo corpo mortale si vesta d’immortalità. }%
\@namedef{54}{Quando poi questo corpo corruttibile si sarà vestito d’incorruttibilità e questo corpo mortale d’immortalità, si compirà la parola della Scrittura: La morte è stata inghiottita nella vittoria. \par}%
\@namedef{55}{\bpoem{}Dov’è, o morte, la tua vittoria? \epoem\bpoem{}Dov’è, o morte, il tuo pungiglione? \epoem\poemsep}%
\@namedef{56}{Il pungiglione della morte è il peccato e la forza del peccato è la Legge. }%
\@namedef{57}{Siano rese grazie a Dio, che ci dà la vittoria per mezzo del Signore nostro Gesù Cristo! \par}%
\@namedef{58}{Perciò, fratelli miei carissimi, rimanete saldi e irremovibili, progredendo sempre più nell’opera del Signore, sapendo che la vostra fatica non è vana nel Signore. \par}%
\endinput
